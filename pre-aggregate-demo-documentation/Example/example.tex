In this section, a concrete example is given to show how to use Stairwalker.
The example runs from creating a pre-aggregation index from a given dataset
to geospacially representing the data using GeoServer. For the example, a
dataset is provided in terms of a .csv file. It concerns an exported list
of tweets sent in the UK and which carry a GPS coordinate.

The result expected in this example can be seen in \Fref{fig:result}.
Different colors represent the number of tweets in a region. The result
only shows the highest granularity. Each tile shows how many tweets are
sent from a location within the tile.  If this layer is combined with a
map, it would be clear that the region of this figure is directly above the
UK.

\subsection{Requirements}
Before showing how to create the given example, all necessary installations
should have been completed and the required files obtained.
The following programs should be installed:
\begin{enumerate}
	\item PostgreSQL (\Fref{sec:postgresql})
	\item PosgGIS for a PostgreSQL database (\Fref{sec:postgis})
	\item Tomcat or another web service server (\Fref{sec:tomcat})
	\item GeoServer with extension (\Fref{sec:geoserverinstall})
\end{enumerate}
Also the following files should be at the ready:\footnote{The files can be
found in the \lstinline|RunningExample| directory in the same Git as this
manual.}
\begin{enumerate}
	\item The tool to make a pre-aggregate index: Pre-Aggregate-Index tool
		(\Fref{sec:preaggtool})
	\item Configuration file for the tool: \lstinline|runningexample.config.xml|
	\item Sample dataset: \lstinline|RunningExample.csv|
	\item Sample SLD file: \lstinline|RunningExamplSLD.xml|
\end{enumerate}
Furthermore it is useful to have a frontend tool in which the PostgreSQL
database can be managed. During development the tool
pgAdmin\footnote{\url{http://www.pgadmin.org/}} was used.

\subsection{Example Table Setup}

The first step of the process is to have a dataset which to be
aggregated. For this example, a dataset is supplied in the form of a
\lstinline|.csv| file. In order to import this, first create the table in
the database. This can be done with the following SQL query:

\subsubsection{Creating the database}

\begin{lstlisting}
CREATE TABLE runningexample (
	id_str character varying(25),
	tweet text,
	user_name text,
	place_name text,
	time timestamp with time zone,
	reply_to text,
	place_id bigint,
	len bigint,
	coordinates geometry,
	CONSTRAINT enforce_dims_coordinates CHECK ((st_ndims(coordinates) = 2)),
	CONSTRAINT enforce_geotype_coordinates CHECK (((geometrytype(coordinates) = 'POINT'::text) OR (coordinates IS NULL))),
	CONSTRAINT enforce_srid_coordinates CHECK ((st_srid(coordinates) = 4326))
);
\end{lstlisting}

Once the table is created, import the \lstinline|.csv| file into the table.

Note that \lstinline|RunningExample.csv| contains column headers, uses
\lstinline|;| as column seperators and \lstinline|"| as quote seperators.
After the import is done, a pre-aggregate index can be created.

\subsubsection{Creating the Pre-Aggregate Index}
\label{sec:examplePreAggIndex}

An in depth discussion of the use of the pre-aggregate index tool is given
in \Fref{sec:preaggtool}. In the example, only commands will be given with
only brief explanations.

First the pre-aggregate index creation tool needs to be compiled, which can
be done with the command below executed in the
\lstinline|pre-aggregate-tools| directory.
\begin{lstlisting}
mvn package appassembler:assemble
\end{lstlisting}
The next step is to put the pre-aggreate tool config file in the
\lstinline|pre-aggregate-tools| directory. Once this is done, the tool can
be called with the following command. Note some variables need to be set
in the listing below. These are
\lstinline|<database>, <host>, <port>, <pass>, <user>|,
which should be filled according to how PostgreSQL was
set up.
\begin{lstlisting}
target\appassembler\bin\create-pa-index -config runningexample.config.xml -d <database> -dbtype postgresql -h <host> -p <port> -password <pass> -s public -u <user>
\end{lstlisting}
This creates a pre-aggregate index of the dataset. In the database
three new tables are created: The pre-aggregate index named
\lstinline|runningexample_pa| and two help tables which keep track of the
indexes and the axes used by those indexes. All the work on the side of the
database is now done, and the next step is to visualize the dataset using
GeoServer.

\subsection{GeoServer Setup}
\label{sec:exampleGeoServer}
\begin{wrapfigure}{r}{0.30\textwidth}
	\centering
	  \vspace{-15pt}
	\includegraphics[width=0.25\textwidth]{Figures/Data.png}
	  \vspace{-10pt}
	\caption{\label{fig:data}Data section of navigator}
	  \vspace{-10pt}
\end{wrapfigure}
This section gives a step by step guide of how to create a visual
geospacial representation using the pre-aggregated index of the example
dataset. This will be done using GeoServer, specifically the GeoServer web
administration interface. This section offers are concrete version of the
deployment discussed in \Fref{sec:deployment}.

\Fref{fig:data} shows the \lstinline|Data| section of the navigator which
can be found on the left hand side of the web administration interface. The
links in this section will be used to navigate between different pages
needed to configure the whole setup.

\subsubsection{Add Source}

\begin{figure}[t]
	\centering
	\subfigure[Add new store \label{fig:newstore}]{
		\raisebox{8mm}{
		\includegraphics[scale=0.6]{Figures/AddNewStores.png} }}
	\subfigure[Select data source type \label{fig:storetype}]{
		\includegraphics[scale=0.6]{Figures/StoreType.png} }
	\caption{Adding new \lstinline|Store| to GeoServer}
\end{figure}

\begin{figure}[p]
	\centering
	\subfigure[\label{fig:createstore}\lstinline|New Vector Data Source|]{
		\includegraphics[width=.4\textwidth]{Figures/CreateStore.png}}
	\subfigure[\label{fig:editlayerdata}\lstinline|Edit Layer| page]{
		\includegraphics[width=.5\textwidth]{Figures/EditLayer_Data.png}}
	\caption{Configuring a store and layers\label{fig:storeandlayers}}
\end{figure}

A data source is added in the following way:
\begin{enumerate}
	\item Click on the \lstinline|Stores| link in the \lstinline|Data|
		section shown in \Fref{fig:data}.
	\item The \lstinline|Stores| will open, the top of the page looks like
		\Fref{fig:newstore}; click on the \lstinline|Add new Store| link.
	\item A selection of different \lstinline|Vector Data Sources| is now
		available. Select \lstinline|NeoGeo Aggregate| as shown in
		\Fref{fig:storetype}. 
\end{enumerate}

\begin{enumerate}[resume]
	\item After selecting \lstinline|NeoGeo Aggregate| as
		\lstinline|Vector Data Source| a page like \Fref{fig:createstore}
		will open. Fill in all the fields as shown. Some values may differ
		depending on how the database is setup. More exact information can
		be found in \Fref{sec:addingsource}.
	\item Once everything is filled out click the \lstinline|Save| button.
		This leads to page where \lstinline|Layers| can be published.
		However before that is done, first the \lstinline|Style| should
		be imported.
\end{enumerate}

\subsubsection{Import Style}

Importing a style is done as follows:
\begin{enumerate}[resume]
	\item Click on the \lstinline|Styles| link in the \lstinline|Data|
		section shown in \Fref{fig:data}.
	\item Click on the \lstinline|Add a new style| button which will go to a
		page similar to \Fref{fig:styleimport} although empty.
	\item Import the \lstinline|RunningExampleSLD.xml| file by using the
		\lstinline|Choose File| button. The \lstinline|Upload...| link
		is highlighted in red in \Fref{fig:styleimport}.
	\item Once the style has been uploaded, the \lstinline|New style| page
		should look like \Fref{fig:styleimport}.
	\item Press the \lstinline|Save| button.
\end{enumerate}
The style used in this example has been imported in GeoServer and now the
layer is ready to published.

\begin{figure}[t]
	\centering
	\includegraphics[scale=0.5]{Figures/NewStyleImport.png}
	\caption{\label{fig:styleimport}Importing SLD style from file}
\end{figure}

\subsubsection{Create Layer}

\begin{figure}[t]
\centering
\includegraphics[width=0.5\textwidth]{Figures/EditLayer_Publishing.png}
\caption{Adding a \lstinline|Style| to the \lstinline|Layer|
\label{fig:layerpublish}}
\end{figure}

\begin{figure}[t]
	\centering
	\includegraphics[width=\textwidth]{Figures/Publishlayer.png}
	\caption{Publishing a \lstinline|Layer|\label{fig:publishlayer}}
\end{figure}

Creating a new layer is done as follows:

\begin{enumerate}[resume]
	\item Click on the \lstinline|Layers| link in the \lstinline|Data|
		section shown in \Fref{fig:data}.
	\item This opens the \lstinline|Layers| page, here click on the
		\lstinline|Add a new resource| button. This open a page similar to
		\Fref{fig:publishlayer}.
	\item Select the \lstinline|Publish| action for the example layer.
	\item A page like \Fref{fig:editlayerdata} will open. Set highlighted
		fields to match \Fref{fig:editlayerdata}. More exact information
		about these fields can be found in \Fref{sec:addinglayers}.
	\item After the fields in the \lstinline|Data| are filled in, go to
		the \lstinline|Publishing| tab, see \Fref{fig:layerpublish}.
	\item Set the default style to \lstinline|RunningExampleSLD| like
		in \Fref{fig:layerpublish}.
	\item Press the \lstinline|Save| button.
\end{enumerate}

A layer for the example dataset has now been created and is ready to be
viewed.

\subsubsection{View Layer}

\begin{figure}[t]
\centering
\includegraphics[width=\textwidth]{Figures/LayerPreview.png}
\caption{Previewing a \lstinline|Layer|\label{fig:preview}}
\end{figure}

The final GeoServer step is to preview the layer. The preview only shows
the highest granularity of the aggregation index. Getting a preview of a
layer is done as follows:
\begin{enumerate}[resume]
	\item Click on the \lstinline|Layer Preview| link in the
		\lstinline|Data| section shown in \Fref{fig:data}.
	\item The \lstinline|Layer Preview| page opens which displays all
		viewable layers like in \Fref{fig:preview}.
	\item A preview format need to be selected from the drop-down menu
		highlighted in \Fref{fig:preview}.
	\item Select a \lstinline|WMS| preview format such as \lstinline|PNG|.
	\item A new web page will load (this might a few seconds depending on
		whether or not the server side extension is enabled).
	\item The final result will look like \Fref{fig:result}.
\end{enumerate}

The layer showing the example dataset is now complete. The values of
each square in the layer is calculated using the pre-aggregate index of the
dataset. See \ref{sec:clientsidedev} to learn how the layer can be used in
combination with other tools such as
OpenLayers\footnote{\url{http://openlayers.org/}} to create a dynamic map
which updates data on the fly.

\begin{figure}[t]
\centering
\includegraphics[width=0.45\textwidth]{Figures/FinalResult.png}
\caption{Preview of whole dataset \label{fig:result}}
\end{figure}

