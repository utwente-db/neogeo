In this section, some pointers will be given to important sections of code
in terms of processing pre-aggregate indexes which are used by the
GeoServer extension.

%MvK: ???
%During the development of Stairwalker, the developers project of GeoServer,
%the source code used is available to download on the GeoServer website.
Using the source code GeoServer and running it from an IDE, provides useful
information for troubleshooting, as logging information is printed directly
to the console, and for constant redeployment a new WAR file does not have
to be remade every time. The source code of the Stairwalker project is
available on GitHib\footnote{\url{https://github.com/utwente-db/neogeo}}.
The two important packages from this project are \lstinline|geoserver-ext|
and \lstinline|pre-aggregate|. The first is the extension used in GeoServer
to handle pre-aggregation indexes as data sources and the second is the
package which is used to create the pre-aggregate index.

\subsubsection{Creating Pre-Aggregate Index from Source}

%\todo[inline, size=\small]{This subsection is still superficial in some aspects. I am unsure how extensive an explanation is required here and if it is even necessary.}

Creating a pre-aggregate index can be done using a tool as shown in
\Fref{sec:preaggtool}. It is also possible to create a new method in the
\lstinline|Test.java| of \lstinline|pre-aggregate| in which the steps for
creating the pre-agggregation index can be done manually. In the
\lstinline|main| of this class configuration file
\lstinline|database.properties| is read, which contains login information
about the database which contains the dataset, and connection is made to
said database. Next a pre-aggregate index is made using the custom made
method which describes the pre-aggregation. How to set up such a method is
described in this subsection.

Firstly for each axis on which the dataset will be split an
\lstinline|AggregateAxis| variable will be defined. Next these variables
should be initiated with a axis type (\lstinline|metric| or
\lstinline|nominal|), in the case of \lstinline|nominal| also a word list
(which contains the words on which the dataset will be split) should be
created. The constructors of both axis types are as follows:

\begin{lstlisting}
public MetricAxis(String columnExpression,
	String type, Object BASEBLOCKSIZE, short N);
	
public NominalAxis(String word_collection_column, 
	String word_index_column, String wordlist_str, 
	String name);
\end{lstlisting}

Once all axes have been initialized, an array containing all axes should be
created which will be used as input for the \lstinline|PreAggregate|
variable created in the next. The next step is to create a pre-aggregate
index by creating and initiating a \lstinline|PreAggregate| variable. The
constructor of \lstinline|PreAggregate| is as follows:

\begin{lstlisting}
public PreAggregate(Connection c, String schema,
	String table, String override_name,
	String label, AggregateAxis axis[],
	String aggregateColumn, String aggregateType,
	int aggregateMask, int axisToSplit,
	long chunkSize, Object[][] newRange);
\end{lstlisting}

Afterwards the connection to the database should be closed. A method
containing these components is created and it can be statically run in the
main. Note that with the \lstinline|NominalAxis|, some prepossessing maybe
required on the dataset. In order to do this, the help method
\lstinline|tagWordIds2Table| is used. On \lstinline|NominalAxis| the
\lstinline|tagWordIds2Table| method has the following arguments:

\begin{lstlisting}
public void tagWordIds2Table(Connection c, 
		String schema, String org_table, 
		String new_table);
\end{lstlisting}

\subsubsection{Filtering Words for Nominal Axis}
\label{sec:filtering}

In the package \lstinline|geoserver-ext|, specifically the class
\lstinline|AggregationFeatureSource| creates the SQL query which requests
data from the pre-aggregated index for which GeoServer is built. If the
pre-aggregated index contains a \lstinline|Nominal| axis, then it is
possible to pass along words to filter the axis in GeoServer using
\lstinline|VIEWPARAMS| (see \Fref{sec:previewlayer}). In order to include
this filter parameter, the method \lstinline|getReaderInternal| and possibly
\lstinline|reformulateQuery| should be extended.

In the method \lstinline|getReaderInternal|, the layer request sent to
GeoServer is parsed including the \lstinline|VIEWPARAMS|, so a variable
should be created in the method which matches the word which should be
filtered given by \lstinline|<TYPE>:<VALUE>|. This variable can then be
passed to the \lstinline|reformulateQuery| method. It is then possible in
\lstinline|reformulateQuery| for a specific \lstinline|AggregateAxis| to
split on the variable passed along in \lstinline|getReaderInternal|.
